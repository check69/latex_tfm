%Conclusion

After reading and summarizing all information I come to the following conclusions:

\begin{itemize}
	\item \textbf{Articles:} 
	The IoT devices have big security flaws, because the majority seek the functionallity and the usability but they do not realize, or do not want to, the necessary actions in order to secure the valuable information of the new products. A lot of papers propose differents measures to develop new devices, but must be the companies the ones who follow them. In this section, I discover a lot of develop techniques, testing methods and new methodologies, all related with IoT field.
	
	The majority of the articles refer to differents methodologies used to test IoT environments, however some articles propose some tools and certificates for IoT testing vulnerabilities. It is very interesting to know all the available methodologies and learn the tools for testing.
	
	\item \textbf{Companies:} I divided them in two groups, because I think that in order to protect IoT device it is important doing it in the two differents ways at the same time. For our work would be convenient talk with some routers companies to integrate our project in their devices, and improve the efficiency.

	\item \textbf{Pen-testing:} In this section, I learned the ethic hacking concept, the vulnerabilities and their common attacks.
	
	\item \textbf{IoT Classification:} I think the biggest market and the most unprotected is smart house. because when a company get a new device, has the required resources to securite it.
\end{itemize}