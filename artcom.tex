%Conclusion

After reading and summarizing all information I come to the following conclusions:

\begin{itemize}
	\item \textbf{Articles:} 
	Los dispositivos IoT tienen fallas de seguridad graves, porque la mayoría busca la funcionalidad y no se han percatado o querido tener en cuenta las medidas necesarias para que la información valiosa de los nuevos dispositivos esté segura. Muchos papers proponen diferentes medidas para desarrollar productos. Pero deben ser las empresas que realizan estas tareas las que deben seguirlos. Uno de los papers aquí expuestos, propone un método para realizarlo una vez el dispositivo esté en el mercado, pudiendo evitar conectarlo si no pasa las pruebas. Creo que este último paper es de suma importancia, porque no solo podemos utilizarlo nosotros como bien dice, sino routers companies para que venga integrado en el proceso de conexión que tenemos en casa. También podrá ser útil para las diferentes empresas que usan dispositivos IoT como herramientas de trabajo.\\
	Otros de los papers expuestos hablan de testing, y de las diferentes formas que han usado para realizar sus pruebas. De los cuales podemos aprender técnicas y realizar los mismos en nuestro trabajo fin de máster.
	
	The majority of the articles refer to differents methodologies used to test IoT environments, however some articles propose some tools and certificates for IoT testing vulnerabilities. It is very interesting to know all the available methodologies and learn the tools for testing.
	
	\item \textbf{Companies:} I divided them in two groups, because I think that in order to protect IoT device it is important doing it in the two differents ways at the same time. Para nuestro trabajo fin de máster sería conveniente hablar con las empresas de routers para introducir nuestro trabajo en ellos de forma nativa y sea eficiente a la hora de evaluar si un dispositivo puede unirse a la red o no.

	\item \textbf{Pen-testing:} En esta sección hemos aprendido el concepto de hacking ético, las vulnerabilidades más importantes y los diferentes ataques a realizar
	
	\item \textbf{IoT Classification:} I think the biggest market and the most unprotected is smart house. because when a company get a new device, has the required resources to securite it.
\end{itemize}