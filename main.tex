\documentclass[a4paper,english,11pt]{article}
% \usepackage[english]{babel}
\usepackage[utf8]{inputenc}
\usepackage{verbatim}
\usepackage{fancyhdr}
\usepackage{lmodern}
\usepackage{tabularx}
\usepackage{multicol}
\usepackage{listing}
\usepackage{adjustbox}
\usepackage{indentfirst}
\usepackage[most]{tcolorbox}
\usepackage{tcolorbox}
\usepackage{lipsum}
%\usepackage{url}
\usepackage[T1]{fontenc}
\usepackage[bookmarks,hyperfootnotes=false,colorlinks=true,urlcolor=black,linkcolor=black,citecolor=black,pagecolor=black,anchorcolor=black,breaklinks=true]{hyperref}% Hiperenlaces

\usepackage{graphicx}
\usepackage{fixme}
\usepackage{lscape}
\usepackage{movie15}
\graphicspath{{images/}}
\DeclareGraphicsExtensions{.png,.pdf,}

% Párrafos
\setlength{\parskip}{8pt}

\begin{document}
	
	\renewcommand*{\thepage}{\roman{page}}
	\newcommand{\IoT}{\textbf{IoT} }
	
	\begin{titlepage}
		\begin{center}
			\vspace*{0,5cm} 
			{\Large \textsc{Universidad Internacional Méndez Pelayo\\}}
			\vspace{3mm}
			{\huge \textbf{Resolución de problemas con metaheurísticos}} \\
			\vspace{1,7cm}
			{\Large \textsc{Testing security Internet of Thing\\}}
			
		\end{center}
		\vspace{12.3cm}
		\begin{table}[!h]
			\Large
			\begin{tabular}{rl}
				\textbf{Author}: & Antonio José Checa Bustos\\
				
			\end{tabular}
		\end{table}
		% \vspace{1,5cm} \\
		% \begin{flushright}
		% {\large \today}
		% \end{flushright}
	\end{titlepage}
	
	
	
	
	\newpage
	\thispagestyle{empty}
	\renewcommand{\contentsname}{Contents}
	\renewcommand{\refname}{References}
	\tableofcontents
	\newpage
	
	\pagestyle{fancy}
	\fancyhead[L]{Testing security IoT}
	% \fancyhead[R]{Antonio José Checa Bustos}
	\fancyfoot[C]{\thepage}
	\renewcommand{\sectionmark}[1]{\markboth{\textbf{\thesection. #1}}{}}
	\renewcommand*{\thepage}{\arabic{page}}
	\setcounter{page}{1}
	
	\section{Introduction}
	\label{sec:intro}
	%Introducción

This work seeks to search relevant information about differents techniques or companies that test the vulnerabilities of the \textbf{Internet of Things (IoT)} devices.

In the making of this work I have been use the following steps:
\begin{itemize}
	\item Searching in SpringerLink and IEEExplore magazine, with a little resume of the differents interesting paper found it.
	\item Looking for companies in this topic, and classified them.
	\item Searching for tutorials and videos for pen-testing with python.
	\item Making a classification of IoT devices availables.
\end{itemize}

For a better understanding of this work, it is precise to define some terms:

\href{http://dictionary.cambridge.org/es/diccionario/ingles/security}{According to the Cambridge Dictionary,} \textbf{security} is the protection of a person, building, organization, or country against threats such as crime or attacks by foreign countries. 

In particular, the concept of \textbf{cibersecurity} has been broadly explain during the last years. Here are some relevant definitions:\cite{Paper0}
\begin{itemize}
	\item ``Cybersecurity consists largely of defensive methods used to detect and thwart would-be intruders.'' (Kemmerer, 2003)
	\item ``Cybersecurity is the collection of tools, policies, security concepts, security safeguards, guidelines, risk management approaches, actions, training, best practices, assurance and technologies that can be used to protect the cyber environment and organization and user's assets.'' (ITU, 2009)
	\item “The body of technologies, processes, practices and response and mitigation measures designed to protect networks, computers, programs and data from attack, damage or unauthorized access so as to ensure confidentiality, integrity and availability.”  (Public Safety Canada, 2014)
	\item “The art of ensuring the existence and continuity of the information society of a nation, guaranteeing and protecting, in Cyberspace, its information, assets and critical infrastructure.” (Canongia \& Mandarino, 2014)
	\item “The state of being protected against the criminal or unauthorized use of electronic data, or the measures taken to achieve this.” (Oxford University Press, 2014)
	\item “The activity or process, ability or capability, or state whereby information and communications systems and the information contained therein are protected from and/or defended against damage, unauthorized use or modification, or exploitation.” (DHS, 2014)
\end{itemize}

In regards to \textbf{Internet of Things:} is important to note the following definition: \textit{``A world where physical objects are seamlessly integrated into the information network, and where the physical objects can become active participants in business process.''}\cite{IoT1}
To  extend  the coverage of IoT definition, \cite{IoT2}it  defines  the  "Things"  from  physical  objects  to  virtual  objects  which represents as the identities with Internet connectivity. Another definition from IEEE IoT initiative \cite{IoT3}defines a ``Thing'' on IoT that indicates a physical or virtual object which connects to the Internet and has the ability to communicate with human users or other objects. 

Summarizing, we can understand IoT as follows:\cite{Paper1}
\begin{itemize}
	\item Internet of Things is essentially an architectural framework which allows integration and data exchange between the physical world and computer systems over existing network infrastructure.
	\item A simple definition of the Internet of Things would be a system of all kinds of network-connected devices that are able to communicate with each other. They can send and receive data from the internet.
\end{itemize}

However, the actual problem is that much of the software installed on IoT devices are speciallized primarily on functionality and has not focus on security, so that brings some serious trouble for people who buy it. If the company who does smart door doesn't focus on security, it  would be easy to break it and get inside.

The main problems about security nowadays are:
\begin{itemize}
	\item It is extremely easy to hack the internet conexion.
	\item Internet service provider (ISP’s) do not offer security for homes. If a family want to increase the security levels, must search an external provider.  
	\item Young people are not protected when day navigate the web
	\item (IoT) is a raising market that do not focus on the security issue. 

\end{itemize}

About the size of the (IoT) market, we have to note that \textit{``Only in 2011 did the number of interconnected devices on the planet overtake the actual number of people. Currently there are 9 billion interconnected devices and it is expected to reach 24 billion devices by 2020. According to the GSMA, this amounts to \$1.3 trillion revenue opportunities for mobile network operators alone spanning vertical segments such as health, automotive, utilities and consumer electronic''}\cite{Data1}

 \includegraphics[width=\linewidth]{iotCountries.png}

Given these information, we can conclude that the (IoT) market is growing at fast pace and has great opportunities to improve. This work will focus on this scope of activity and has the following structure:
\begin{enumerate}
	\item The articles are summarized in paragraphs in order of publishing in section 2.
	\item In Section 3 there are the differents companies that handles the IoT security.
	\item In section 4 there are some links for pentesting and of IoT devices lists.
	\item In section 5 there is a classification of IoT devices.
	\item In section 6 there is a little conclusion of this research.
\end{enumerate}	
	
	\newpage
	
	\section{Articles}
	\label{sec:articulos}
	%Articulos

\subsection{Resumen de los diferentes Artículos encontrados}

\subsubsection{How to Test IoT-based Services before Deploying them into Real World}\cite{Paper2}
This paper describes  a  new  approach  for  testing  IoT-based  service  build on  a  code  insertion  methodology,  which  can  be  derived  from the semantic description of the IoT-based service.
The proposed IoT resource emulation interface is described from the semantic, architectural  and  implementation  perspective.  The  paper  compares its applicability and efficiency with classical approaches and expose  high  emulation  capabilities  while  minimising  the  testing effort.
They propose a semiautomated process of test code insertion in order to support fast prototyping with test integration of large-scale IoT-based services, in other words, they propose a code insertion methodology, because \IoT is so heterogenous there is very difficult to test every device with one code.
The paper describes an example how a stream X-Machine model can be derived from service interface descriptions based on on-tologies and rules.

\subsubsection{SWE Simulation and Testing for the \IoT}\cite{Paper3}
This paper presents the use of a high level sensor simulator and sensor web standards applied to an indoor industrial environment, in order to target worker's safety. In order to achieve this goal they developed a simulator representing multiple types and number of sensors and integrate it with a standard sensor database known as Sensor Observation Service (SOS). In addition to that, they develop a Human Computer Interface (HMI) to monitor all sensors and also a Control Center application to test one of the risks within the factory (collision detection).\\
The Wireless Sensor Network, SOS and Control center are communicated by network. Also, they have sensor emulator that may coexist in the environment. The HMI (Human Computer Interface) is responsible for offering a web monitoring platform, as will be described later.\\
The simulator replaces an individual or a group of sensors, and the key communication component at application level is the interaction with the SOS. \\
The simulator is configured through an XML file that includes a list of all sensors to be simulated with their description and specific configuration.\\
The simulator presented in the paper is able to generate realistic and real time data from both mobile and fixed sensors, sending all measurements to a Sensor Observation Service (SOS) which aggregates all sensor data as a unified entity. \\
We can learn from this simulator to emulate some different device and test \IoT.\\

\subsubsection{Internet of Things (IoT): A vision, architectural elements, and future directions}\cite{Paper9}
This paper presents the current trends in IoT research propelled by applications and the need for convergence in several interdisciplinary technologies. It presents a taxonomy that will aid in defining the components required for the Internet of Things from a high level perspective. It also shows a list of several application domains which will be impacted by the emerging Internet of Things. They can be classified based on the type of network availability, coverage, scale, heterogeneity, repeatability, user involment and impact.They categorize the application into four application domains: Personal and Home, Enterprize, Utilities and Mobile.\\
They propose a framework enabled by a scalable cloud to provide the capacity to utilize the IoT. In proposing the new framework associated challenges have been highlighted ranging from appropriate interpretation and visualization of the vast amounts of data, through to the privacy, security and data management issues that must underpin such a platform in order for it to be genuinely viable.

\subsubsection{Internet of Things: Architectural Framework	for eHealth Security}\cite{Paper8}
The Internet of Things (IoT) holds big promises for Health Care, especially in Proactive Personal eHealth.\\
This paper defines eHealth from differents sources because it is not a single consensus definition. The World Health Organization [WHO] defines e-Health as:\\
``E-health is the transfer of health resources and health care by electronic means''\\
They propose an architecture and benefits and join both worlds. It explains which are the security problems making emphasis in a IoT device is not a safe device. It explain very well each critical point:
\begin{enumerate}
	\item Security in the IoT landscape
	\item Endpoint device security
	\item Network security
	\item Cloud-based application access security
	\item Data storge security
	\item Enterprise application access security
	\item Federated secure access to partner cloud service
\end{enumerate}

Finally they conclude that the future of eHealth is related to a succesful deployment of a secure and privacy-preserving M2M/IoT infrastructure. 

\subsubsection{Testing protocols in Internet of Things by a formal passive technique}\cite{Paper11}
The main objective of this paper is propose a passive distributed testing approach, for testing both the conformance and the performance of Extensible Messaging and Presence Protocol (XMPP) under IoT environment, based on their formal testing technique. This work define a formalism to specify conformance and performance requirements of XMPP represented as formulas tested on real protocol traces. And then, since several protocol requirements need to be tested on different wireless entities, they design a distributed framework for testing their approach on runtime wireless network execution traces.\\
The main contributions of this work are: A formal approach for formally specifying conformance and performance requirements of XMPP, a convergence of testing conformance and performance together with the same approach and a distributed testing framework designed for testing XMPP services in IoT environment.

\subsubsection{IoT Attack Surface Mapping}\cite{Presentation1}
This presentation shows the differents attacks that may occur on IoT devices. It explain which are the areas that can be vulnerable not only IoT but any device. We explain in more detail the attacks to this areas in section \ref{sec:pentesting}.

These are the vulnerable parts of any device, some of them those no have to be in the device. For example if the device has not a web interface, for this device, this vulnerability disappear.
\begin{itemize}
	\item Ecosystem Access Control
	\item Device Memory
	\item Device Physical Interfaces
	\item Device Web Interface
	\item Device Firmware
	\item Device Network Services
	\item Administrative Interface
	\item Local Data Storage
	\item Cloud Web Interface
	\item Ecosystem Communication
	\item Vendor Backend APIs
	\item Third-Party Backend APIs
	\item Update Mechanism
	\item Mobile Application
	\item Vendor Backend APIs
	\item Network Traffic
\end{itemize}

\subsection{Internet of things laboratory test bed.}\cite{Paper12}
The article describes the structure and composition of the laboratory test bench and partially—laboratory equipment for testing and prototyping.\\
The stand consists of three segments: data collection, local positioning, and data storage and processing.\\
A brief description of approaches to testing of wireless sensor networks based on technology ZigBee: Testing IEEE 802.15.4 (PHY/MAC Layer), testing the recovery wireless sensor networks, network testing for interaction with public communication networks.\\
The potential of the use of the Internet of Things Laboratory for learning is mentioned. Some of the projects that were developed by students and masters Petersburg SUT are enumerated.\\

\subsubsection{IoT: Source of test challenges}\cite{Paper4}
This paper provides seven diferent point of view to analyze the \IoT devices, just like its challenges, solutions and asocciated trade-offs for testing.
\begin{enumerate}
	\item There are three traditional architectures supported within the IoT space:
	\begin{itemize}
		\item ``High End'' Edge Device typically leverage mobile-phone architectures.
		\item ``Low End'' Edge Device are typically microcontroller-based architectures with embedded Flash or lower-cost NVM options.
		\item The other ``Low End'' Edge Device architecture is the trend of adding `intelligence'
	\end{itemize}
	This point of view also have focus in security, heterogeneous test integration and per-unit price.
	
	\item Mario Konijnenburg describes that low cost is the main driver in the development of new designs and technologies for IoT. The costs of testing these IoT design is relatively very high and might not be needed when spare sensor nodes are used in the sensoric swarm. Also say that built-in self-test is effective but complex.
	
	\item  Chih-Tsun Huang and Ping-Hsuan Hsieh put the focus in those IoT devices that have a low power and Energy harvesting. They realize that is a challenge make tests for these devices.
	
	\item This section reviews aspects of the different test challenges and proposes solutions to answer them in a cost-effective way. 
	
	\item Lightweight high quality testing of True Random Number Generators and Physically Unclonable Functions is absolutely required to exploit the full potential of these building blocks. Without these tests, non-intentional failures or dedicated attacks are not detected, and hence pose significant security hazards to IoT devices.
	
	\item This section describes Route of Trust for IoT Manufacturing. Inserting secret keys, tokens, certificates, and boot loaders into the device during wafer probe or ultimately at final test on package level is a sensitive operation.
	
	\item This section talk about testing costs and how to decrease that cost with almost a factor of 100.\\
\end{enumerate}

\subsubsection{Learning Internet-of-Things Security ``Hands-On''}\cite{Paper10}
The fast productization of IoT technologies is leaving users vulnerable to security and privacy risks.\\
This paper shows the security implication of IoT security.
Este paper habla de las implicaciones de securidad de IoT. They didn't attempt to provide wide coverage but rather to highlight some of the most severe, yet easy to abuse, security and privacy threats to simple IoT use cases, namely:
\begin{itemize}
	\item Leakage of personally identifiable information (PII): In this section they explain that IoT devices have a lot of personal information, and also confirm that the risk may be identified, assure the communication and authentication, as well as hide the identity.
	
	\item Leakage of sensitive user information: There are several possible reasons that explain that applications commonly collect redundant data or data not directly relevant to their purpose. First, application developers might overestimate the requirements of future, improved application versions. Second, applications often don't communicate properly with their users regarding the type of sensitive information being collected or don't provide opt-out options. Finally, the flow of sensitive information is sometimes the outcome of bad protection practices during data transmission (for example, not using Transport Layer Security). In this case, anyone capable of eavesdropping on the communication might access the information exchanged.
	
	\item Unauthorized execution of functions: Today, attackers have many opportunities to infiltrate a network: malware might evade antivirus checks, mobile apps with back doors might allow remote code execution, vulnerable services running on clients might allow buffer overflow attacks, and unnecessarily open ports might welcome unauthenticated malicious entities. Given these risks and modern networks' increasing complexity, it's unrealistic to assume that the exposed clients are reliable and trustworthy. In fact, a compromised client inside the network is often used as a stepping stone for an unauthorized entity outside the network to issue commands that affect the status of local devices. The threat is significant for the IoT ecosystem because IoT devices interact with the physical world and users. Surprisingly, their use case analysis indicates that some IoT products adopt insecure mechanisms by default to provide a more user-friendly plug-and-play product.
\end{itemize}

\subsubsection{A Review on Big Data Analysis and Internet of Things}\cite{Paper5}
This paper aims at reviewing the role of big data in IoT through discussion of its protocols and architectural structure. Also its presents methods for testing these protocols and security considerations. Multiple IoT applications and future research directions are also discussed.\\
Explains the importance of big data in IoT and discusses various challenges associated with its management and communication. Clarifies different mode of communication between IoT for big data processing while overviews its layered structure architecture. Describe different platforms to test designed protocols for managing IoT, discuss the importance of security and privacy of IoT is completed. The possible applications of IoT in different areas is explained, and future research challenges are discussed. \\
This paper also provides testing tools for new and old protocols, like Net2Plan, EuWIn SensorHUB.
This is an interesting paper where we can use for differents protocols testings.

\subsubsection{ARMOUR: Large-scale experiments for IoT security \& trust}\cite{Paper6}
This paper presents ARMOUR, a research project in which a methodology to experiment, validate and certify different technological solutions in large-scale conditions is defined. Additionally, a set of bootstrapping, group sharing and software programming experiments is proposed, on which different tests will be executed with the purpose to verify their security and trust in IoT scenarios.\\
The methodology consist of four phases:
\begin{itemize}
	\item Experimentation Definition \& Support: Definition of the IoT security and trust experiments; Research and development of the ARMOUR technological experimentation suite and benchmarking methodology for executing, managing and benchmarking large-scale security and trust IoT experiments; Analysis of the FIT loT-LAB testbed and FIESTA IoT/Cloud platform for evaluating their composition, supports and services from the perspective of ARMOUR experimentation.
	
	\item Testbeds Preparation \& Experimentation Set-up: Extend, adapt and configure the FIT IoT-LAB testbed with the ARMOUR experimentation suite to enable IoT large-scale security and trust experiments from FIT IoT-LAB and FIESTA IoT/Cloud testbeds;	Prepare the FIESTA IoT/CIoud platform to hold data set from ARMOUR experiments enabling researchers to perform security and trust oriented experiments, generate new datasets and perform benchmark; Setting-up and preparing the ARMOUR experiments by specifying the security and trust test patterns for the experimentation that will be used to execute and to manage such experiments.
	
	\item Experiments Execution, Analysis and Benchmark: Configure; Measure; Pre-process; Analyse; Report
	
	\item Certification/Labelling \& Applications Framework: Develop a new labelling scheme for large-scale IoT security and trust solutions that provides the user and market confidence needed on their deployment, adoption and use; Define a framework to specify how the different security and trust solutions can be used to support the design and deployment of secure and trusted applications for large-scale IoT.
\end{itemize}
This paper defines the experimentation methodology of ARMOUR project in order to test and validate various technologies solutions applied in determined experiments, particularly in bootstrapping, group sharing and software programming experiments, from the security and trust perspective. Thus, these solutions can be used in different application domains, like Remote Healthcare or Smart Cities.
This methodology would be useful for us in case we need to validate our project.

\subsubsection{Toward consumer-friendly security in smart environments}\cite{Paper7}
This paper presents an implementation of an IoT architectural framework based on Software Defined Networking (SDN). In this architecture, IoT devices attempting to join an IoT network are scanned for vulnerabilities using custom vulnerability scanners and penetration testing tools before being allowed to communicate with any other device. In the case that a vulnerability is detected, the system will try to fix the vulnerability. If the fix fails, then the user will be alerted to the vulnerability and provided with suggestions for fixing it before it will be allowed to join the network.\\
According to OWASP, the following are some of the most common IoT vulnerabilities. They have developed or are developing scanners that focus on each of these items.
\begin{itemize}
	\item Insufficient Authentication/Authorization
	\item Insecure Network Services
	\item Lack of Transport Encryption
	\item Insufficient  Security  Configurability
	\item Insecure Software/Firmware
\end{itemize}
This paper is very similar to our project, because it makes tests for the IoT Devices, find vulnerabilities and also try to fix them.

\subsection{Tabla cronológica}

\begin{center}
	\begin{adjustbox}{max width = \textwidth}
		\begin{tabular}{|c|c|c|c|} \hline
			Date & Reference     & Description                          & Importance \\ \hline
 			2013 & \cite{Paper2} & Propose a code insertion methodology & Learn differents methodologies\\ \hline
			2013 & \cite{Paper3} & 
				\begin{tabular}{@{}c@{}}
					Sensor simulator and sensor web standards\\
					applied to an indoor industrial environment 
				\end{tabular}
										   & Emulate some different devices\\ \hline
		     2013 & \cite{Paper9} & Trends in IoT & 
		     \begin{tabular}{@{}c@{}}
		     	Taxonomy of components required\\
		     	for Internet of things 
		     \end{tabular} \\ \hline
			 2014 & \cite{Paper8} & e-Health and IoT how to connect & 
				 \begin{tabular}{@{}c@{}}
					The security to improve\\
					in devices development\\
					that are very important\\
					for health.
				 \end{tabular} \\ \hline
			 2014 & \cite{Paper11} & Testing in IoT & A different way to test IoT\\ \hline
		     2015 & \cite{Presentation1} & 
		     \begin{tabular}{@{}c@{}}
		     	A brief guide to review\\
		     	the vulnerabilities areas\\
		     	of the IoT devices.
		     \end{tabular}
	      & The vulnerable spots of devices\\ \hline
			 2016 & \cite{Paper4} &
			 \begin{tabular}{@{}c@{}}
			 	Provides seven different point of view\\
			 	to analyze IoT devices. 
			 \end{tabular} 
			 	 & Generic vision of a IoT product.\\ \hline
		 	 2016 & \cite{Paper12} &  & \\ \hline
		 	 2016 & \cite{Paper10} & IoT security & New division of vulnerabilities\\ \hline
			 2017 & \cite{Paper5} & Role of big data in IoT devices &
			 \begin{tabular}{@{}c@{}} 
			 	Troubles and solution of \\introduce big data in IoT 
			 \end{tabular}\\ \hline
			 2017 & \cite{Paper6} & How to develop a larg-scale project & Validate a technology\\ \hline
			 2017 & \cite{Paper7} & 
			 \begin{tabular}{@{}c@{}}
			 	Implementation of an\\
			 	IoT architectural framework\\
			 	to test IoT before we use
			 \end{tabular}  & Important architecture for our project\\ \hline
		\end{tabular}
	\end{adjustbox}
\end{center}

\subsection{Conclusiones}
One of the papers explained here, propose a method to perform once the device is on the market, so we can avoid give connectivity to the device if the test fail. This paper is very important, because is similar to our project and for router companies explain in section \ref{sec:empresas} to integrate in conexion process as the article explain.\\
The other papers talk about testing and many differents ways that they use to apply in many IoT devices.\\
Some papers shows new methodologies and try to teach companies how must be a safe development.

%Explica que como los dipositivos \IoT necesitan muchas conexiones cortas, Un algoritmo como LWC provee la seguridad deseada ya que una nueva clave será generada por sesión, así, el atacante no tendrá suficiente tiempo para romper la clave. La clave es generada por los tweets de miles de personas, lo que hace que sea imposible descifrar la aleatoriedad. Dicen que proporciona un nivel de seguridad bastante elevado y que el overhead es desdeñable.
%Pratham Majumder, Koushik Sinha, \href{http://ieeexplore.ieee.org/document/7917638/authors}{A novel key generation algorithm from twitter data stream for secure communication in \IoT} (13-17 de Marzo de 2017) 4 mayo de 2017
%
%
%
%Este artículo es super interesante para los dispositvos \IoT ya que explica como se puede meter una contramedida hardware para evitar un tipo de ataque, aunque para el proyecto no sirve, pero para tecteco si.
%Ryota Jinnai, Atsuo Inomata, Ismail Arai, Kazutoshi Fujikawa, \href{http://ieeexplore.ieee.org/stamp/stamp.jsp?tp=&arnumber=7917533}{Proposal of Hardware Device Model for IoT Endpoint Security and its implementation} (13-17 marzo 2017) añadido el 4 mayo
%
%
%Hichem Sedjelmaci, Sidi-mohammed Senouci, Tarik Taleb, \href{http://ieeexplore.ieee.org/document/7920414/authors}{An Accurate Security Game for Low-Resource IoT Devices} (5 Mayo 2017)
%
%
%
%
%
%
%
%Reetz, E.S., Kümper, D., Lehmann, A., Tönjes, R.: Test Driven Life Cycle Management
%for Internet of Things based Services: A Semantic Approach. In: The Fourth International
%Conference on Advances in System Testing and Validation Lifecycle (VALID 2012),
%pp. 21–27 (2012)
%De Nicola, R., & Hennessy, M. C. (1984). Testing equivalences for processes. Theoretical computer science, 34(1-2), 83-133.
%
%Boyd, J. H., & Runkle, D. E. (1993). Size and performance of banking firms: Testing the predictions of theory. Journal of monetary economics, 31(1), 47-67.	
	
	\newpage
	
	\section{Companies}
	\label{sec:empresas}
	%Empresas


The majority of the companies that provides security for IoT, are focused on offer solutions only for bussiness, regardless if they have conected devices or if they are involved with the device lifecycle. This way, the companies has influence in their develop, discovering its vulnerabilities from code and testing.

The bussiness focused on \IoT security can be classified in two groups. 

\subsection{Router companies}

Seek the network security by avoiding that any outsider user can connect and enter to the devices(routers).

This business are classified in two groups:
\begin{enumerate}
	\item \textbf{Domestic field:} They care much about the image and communication of its business, trying to make the customer forget about the typical old image of routers and sell it as an object of decoration that aim the function of protect or give more powerful wifi without interruptions according to each case. The aesthetic design of the router becomes a fundamental aspect in this type of companies and they give less information about the technical characteristics of the product. They opt for a very visual and advertising approach, with promotional videos that families can quickly empathize with. They tend to sell speed and comfort, giving importance to the ease of use in many cases from a single device (usually have a mobile application to control everything from there). They also sell protection especially for children (parental control). It is clear that they want to reach parents who are concerned about the safety of their children and the electronic devices in their home. They often use close and direct messages with understandable language that arouses feelings of protection of home and family.
	
	\item \textbf{Business field:} They focus on giving a more serious and technical approach. They are less attractive visually and focus on giving technical information of what their products offer, almost like an anti-virus brand. In their website, they usually have a section where they tell stories of customers (companies, institutions, organizations, etc.) that use their services or products and all sectors where they operate. They offer a list with specific solutions for each case and usually add a section with a list of all the technical characteristics they meet (sometimes compared to other types of services). They do not want to evoke emotions in their messages as in the domestic oriented companies. We find a much more technical and informative language.
	\\
	
	A list of some representative companies is shown below:
	
\end{enumerate}
\begin{itemize}
	\item \href{https://tecteco.com/}{\textbf{Tecteco Security System S. L.}} Seeks to secure the entire environment from the access point, instead of trying to verify that all devices are safe. Tecteco security prevents access to the device, because if you are not in the network you will not be able to access the device.\\
	\textbf {Target:} Medium to small business and families.\\
	\textbf{Keywords:} security, better, conectivity, solutions\\
	\begin{center}
		\includegraphics[width=\linewidth,height=4cm,keepaspectratio]{tecteco.jpg}
	\end{center}
	
	\item \href{https://on.google.com/hub/}{\textbf{Google onHub}} is Google's router. They sell speed and quality of wifi connection with elegant design. Focused on streaming viewing and content downloads on all types of devices, for domestic use also. The concept of security does not appear. They assure to be the first router that offers compatibility with IFTTT. They have an image very aesthetic, with very detailed images. They give relevance to the design of the router (elegant, modern, minimalist).\\ 
	\textbf{Target:} Small to medium companies and families\\
	\textbf{Keywords:} better, faster, easier, simplicity, conectivity\\
	\begin{center}
		\includegraphics[width=\linewidth,height=5cm,keepaspectratio]{onhub.png}
	\end{center}
	
	\item \href{https://www.getcujo.com/}{\textbf{Cujo}} They sell home protection against hackers, especially security related to minors as parental control. They also offer webcams protection and focus on privacy (They do not sell speed or convenience but simplicity). They sell ease of use (connected to your router)\\
	\textbf{Target:} Families or single users\\
	\textbf{Keywords:} Smart way, smart firewall, home hacking, cyber threats, private online, ``easy target from home hackers'', security, simple, internet of things.\\
	\begin{center}
		\includegraphics[width=\linewidth,height=5cm,keepaspectratio]{cujo.jpg}
	\end{center}
	
	\item \href{https://mytorch.com/}{\textbf{Torch}} They sell security for families. As soon as we enter the website, they announce that they have been successful in raising kickstarters. Parental control.\\
	\textbf{Target:} Families (parents)\\
	\textbf{Keywords:} Smart, families, wifi, router, bedtime for each child, filtering, reporting, visibility, guide, balance, “better internet”, parental controls, intuitive experience, ultrafast, powerful.\\
	\begin{center}
		\includegraphics[width=\linewidth,height=5cm,keepaspectratio]{torch.jpg}
	\end{center}
	\newpage
	\item \href{https://omnia.turris.cz/en/#features}{\textbf{Turris Omnia}} They sell security to control all the devices of your house in a very technical way. They use overall poor pictures and communication. Similar to the traditional router companies.\\
	\textbf{Target:} Expert or frecuent computer user of about 30 years old.\\
	\textbf{Keywords:} security, speed, devices, powerful.\\
	\begin{center}
		\includegraphics[width=\linewidth,height=5cm,keepaspectratio]{turrisOmnia.jpg}
	\end{center}
	
	\item \href{https://www.eero.com/}{\textbf{Eero}} They sell wifi speed and security, and assure a wifi that never fails and you do not have to "reset". They offer the posibility of install several wifi devices at home according to their size and recommend where to put them through an app. They focus on comfort and useful design for corporative and domestic use. They also sell security and protection of personal data and the posibility of prevent cyber attacks. They are sold as "the ultimate in security". \\
	\textbf{Keywords:} reliability, impact, speedy, security\\
	\begin{center}
		\includegraphics[width=\linewidth,height=5cm,keepaspectratio]{eero.png}
	\end{center}
\newpage
	\item \href{https://getluma.com/}{\textbf{Luma}} They sell minor surveillance and strong parental control. They include a filter to block some sites (the users can send filter unlock requests for parents to rate with chat window). The router are sold in packages of three. Just one administrator manage the system and can stablish time connection limits. Their website includes a broad section about its creators and who are the investors.\\
		\textbf{Target:} Specially oriented to parents and small companies.\\
	\textbf{Keywords:} fastest, simplest, most secure, most reliable, safety, inteligent, family friendly.\\
	\begin{center}
		\includegraphics[width=\linewidth,height=5cm,keepaspectratio]{luma.jpg}
	\end{center}

	\item \href{http://www.aerohive.com/products/routers/br100.html}{\textbf{AeroHive}} They sell specific solutions for each industry that includes cloud networking, great Wi-Fi, applications \& Insights.\\
    \textbf{Target:} Companies and organizations (schools, energy companies, etc.) In their website they include a section with solutions for each case: (education, enterprise, healthcare, manufacturing \& distribution, retail, state and government.)\\
	\textbf{Keywords:} Unified wired, Wireless, networking, insights, self-organizing access points, effortlessly grow from a single access point, cloud-based management platform.\\
	\begin{center}
		\includegraphics[width=\linewidth,height=5cm,keepaspectratio]{aerohive.png}
	\end{center}
\newpage	
	\item \href{https://www.ruckuswireless.com/products/system-management-control/unleashed}{\textbf{Ruckus Unleashed}} They sell higher performance for small businesses with less costs, and promise to help them develop their business with a better customer service experience. They highlight usability and speed of use: ``deploy Wi-Fi in minutes and configure it in just 60 seconds!'' ``Ruckus offers an easy migration path to controller based Wi-Fi, using the same Wi-Fi access points''\\
	\textbf{Target:} custom-designed to help small business owners grow their business, deliver an excellent customer experience and manage costs while supporting Wi-Fi and a variety of mobile devices with minimal IT staff.\\
	\textbf{Keywords:} built-in controller capabilities, user access controls, guest networking functions, advanced Wi-Fi security, traffic management, small business environments, superior performance, lower costs, simplified management, reducing up-front costs, simplified web interface.\\
	\begin{center}
		\includegraphics[width=\linewidth,height=5cm,keepaspectratio]{ruckus.jpg}
	\end{center}
	
	\item \textbf{Los proveedores de servicio de internet (ISP)} Also try to help the houses security, but very lightly for now.
\end{itemize}

\subsection{Security companies} 

This companies seek to help with the development of a safe product and provide tools or personnel to check and change at any time any possible vulnerability.
\begin{itemize}
	\item \textbf{\href{http://www.beyondsecurity.com}{Beyond security}}
	provides several testing solutions for assess and manage security weaknesses in networks, applications, industrial systems and networked software. They simplify the network management and application security in order to reduce attacks and data loss, meeting security policy requirements and industry and government standards. It is focused only to bussiness. 
	
	They have two products:
	
	\begin{itemize}
		\item \textbf{\href{http://www.beyondsecurity.com/avds.html}{AVDS}} Traditional VA/VM systems tend to be error-prone and can generate bloated, hard to use reports. AVDS with its zero false positive design will identify vulnerabilities with precision so that you can act with certainty to eliminate your network's most serious security weaknesses. At the same time you may discover that your network doesn't actually have many of the issues being reported by your current scanning solution.
		
		\item \textbf{\href{http://www.beyondsecurity.com/bestorm.html}{beSTORM}} is a commercial, black box, intelligent fuzzer. It is used in a lab environment to test application security during development or to certify software and networked hardware prior to deployment. It does dynamic security testing of products in development and can be used by network administrators to certify the security of networked applications before deployment. Software QA departments that may be using a dozen different tools to test application security can now get all dynamic security testing done with just one. Administrators who must certify applications before deployment can now use one tool to test all networked applications - even those with proprietary protocols. 		
	\end{itemize}
	
	\item \href{https://www.praetorian.com/}{\textbf{Praetorian.com}} Offers solutions for clients to find, fix, stop, and ultimately solve cybersecurity problems across their entire enterprise and product portfolios.\\
	For \IoT, it provides coverage across technological domains, including embedded devices, firmware, wireless communication protocols, web and mobile applications, cloud services and APIs, and back-end network infrastructure.

	\item \href{https://aws.amazon.com/es/iot/?sc_channel=PS&sc_campaign=acquisition_ES&sc_publisher=google&sc_medium=english_iot_nb&sc_content=iot_p&sc_detail=iot&sc_category=iot&sc_segment=153188699025&sc_matchtype=p&sc_country=ES&s_kwcid=AL!4422!3!153188699025!p!!g!!iot&ef_id=V1R0KQAABC@8IbS5:20170506155609:s}{\textbf{Aws.amazon.com}}
	Provides three different services: AWS Greengrass is software that lets run local compute, messaging \& data caching for connected devices in a secure way; AWS IoT is a managed cloud platform that lets connected devices easily and securely interact with cloud applications and other devices and the AWS IoT Button is a programmable button based on the Amazon Dash Button hardware. 
\end{itemize}
	
	\newpage
	
	\section{PenTesting}
	\label{sec:pentesting}
	%pen testing

To test the vulnerabilities of a device, we have to attack or hack it.

The term ``hacker'' has a double usage in the computer industry (Palmer, 2001). Originally, the term was defined as:  
\begin{enumerate}
	\item A person who learns the particular details of computer systems and the possible ways to extend their abilities, unlike  the  most  users  of  computers,  who  are  learning  only the minimum skills necessary.  
	\item A person who programs devotedly or who enjoys programming instead of only theorizing about programming.
\end{enumerate}

The definition of ethical hacking (SecuritySearch, 2006) is  ``a computer or network expert who attacks a system on behalf of its owners, seeking vulnerabilities that a malicious hacker can exploit''. Ethical hacking is one of the growing areas of ethics and Information Security. The term ``ethical hacker'' has become popularized by corporations hiring security professionals to test their systems for vulnerabilities and describing these individuals as ``ethical hackers''.\cite{ethicalHacking}

The essential terminology is:

\begin{itemize}
	\item \textbf{Hack Value:} Hack value is the notion among hackers that something is worth doing or is interesting.
	\item \textbf{Exploit:} To, in some way, take advantage of a vulnerability in a system in the pursuit or achievement of some objective. All vulnerability exploitations are attacks but not all attacks exploit vulnerabilities.\cite{Hacking1}
	\item \textbf{Vulnerability:} Hardware, firmware, or software flow that leaves an AIS open for potential exploitation. A weakness in automated system security procedures, administrative controls, physical layout, internal controls, and so forth, that could be exploited by a threat to gain unauthorized access to information or disrupt critical processing
	\item \textbf{Target of Evaluation:} is an IT system, product or component that is identified/subjected to a required security evaluation.
	\item \textbf{Zero-day Attack:} The attacker exploits the vulnerabilities in the computer application before the software developoer releases a patch for them.
	\item \textbf{Daisy Chaining:} Attacker who get away with database theft usually complete their task and the backtrack to cover their tracks by destroying logs, etc.
\end{itemize}

Information Security Threats are broadly classified into three categories, as follows:
\begin{enumerate}
	\item Natural threats include natural disaster such as earthquakes, hurricanes, floods, or any nature-created disaster that cannot be stop.
	\item Psysical threats may include loss or damage of system resources through fire, water, theft, and physical imact.
	\item Human threats can be further classified into three types, as follows.
	\begin{enumerate}
		\item \textbf{Network threats}: A malicious person may brek into the communication channel and steal the information traveling over the network.\\
		The attacker can impose various threats on a target network:
		\begin{itemize}
			\item Information gathering: Through social engineering the enemy try to find ways to get the credentials to steal the data. One way of gather information is by footprinting which can be both passive and active. Reviewing the company's website is an example of passive footprinting, whereas calling the help desk and attempting to social engineering them out of privileged information is an example of active information gathering.\cite{Hacking3}
			\item Sniffing and eavesdropping: It is a technique by which you can "hear" everything that circulates through a network.
			\item Spoofing: It is about try to supplant the user identity and access the system.
			\item Session hijacking: Sometimes also known as cookie hijacking is the exploitation of a valid computer session—sometimes also called a session key—to gain unauthorized access to information or services in a computer system.
			\item main-in-the-middle attacks: Is an attack that acquires the ability to read, insert and modify at will, the messages between two parties without any of them knowing that the link between them has been violated.
			\item SQL injection: This attack inserts SQL code in order to alter the operation of the program.
			\item ARP poisoning: Is a technique by which an attacker sends Address Resolution Protocol (ARP) messages onto a local area network. Generally, the aim is to associate the attacker's MAC address with the IP address of another host, such as the default gateway, causing any traffic meant for that IP address to be sent to the attacker instead.
			\item Password-based attacks: This technique involves trying to figure out the password of a system to access it. The most commonly used variants are brute force attacks and dictionary attacks.
			\item Denial of service attack: This is an Attack on a computer system or network that causes a service or resource to be inaccessible.
			\item compromised-key attack: Occurs when the attacker determines the key, which is a secret code or number used to encrypt, decrypt, or validate secret information
		\end{itemize}
		\item \textbf{Host threats} are directed at a particular system on which valuable information resides.\\
		The following are possible threats to the hosts:
		\begin{itemize}
			\item Malware attacks: Software whose purpose is to damage the system to which it accedes.
			\item Target Footprinting: It allows a hacker to gain information about the target system.
			\item Password attacks: The same technique as the one used in the password-bassed attacks on networks. 
			\item Arbitrary code execution: SQL injection is one of the types of attack.
			\item Unauthorized access: Is the act of gaining access to a network, system, application or other resource without permission.
			\item Privilege escalation: Is a type of network intrusion that takes advantage of programming errors or design flaws to grant the attacker elevated access to the network and its associated data and applications. 
			\item Back door attacks: Is a means of access to a computer program that bypasses security mechanisms.
			\item Physical security threats: A malicious agent can damage computers by physical attacks, such as setting them on fire or breaking them.
			\item Denial of service attacks: Attack on a computer system or network that causes a service or resource to be inaccessible.
		\end{itemize}
	\newpage
		\item \textbf{Application} might be vulnerable to different types of application attacks.\\
		The following are possible threats to the application:
		\begin{itemize}
			\item Data/input validation: Data entry into the application may be compromised.
			\item Authentication and Authorization attacks: Authentication and authorization of the application can be sniffing or spoofing.
			\item Configuration management: The application may have security holes in its configuration files
			\item Information disclosure: This type of attack is aimed at acquiring system specific information about a web site including software distribution, version numbers, and patch levels
			\item Session management issues: Session hijacking.
			\item Buffer overflow issues: Occurs when a program exceeds the amount of memory allocated, so arbitrary code can be executed.
			\item Cryptography attacks: Is a method of circumventing the security of a cryptographic system.
			\item Parameter tampering: Is a form of Web-based attack in which certain parameters in the Uniform Resource Locator (URL) or Web page form field data entered by a user are changed without that user's authorization.
			\item Improper error handling and exception management: If errors or exceptions are not handled properly they can provide valuable information to attackers.
			\item Auditing and loggin issues: Another way to get the information needed to attack the system.
		\end{itemize}
	\end{enumerate}
\end{enumerate}


%It has been searched differents links for pentesting task:
%\begin{itemize}
%	\item \href{https://www.youtube.com/watch?v=zUok5HeZGyA}{IoT penetration for ZigBee}
%	\item \href{http://blog.attify.com/2016/06/30/guide-to-iot-pentesting/}{Guide to pentesting IoT}
%	\item \href{https://www.securityartwork.es/2014/05/13/python-para-el-pentest-introduccion/}{Python pentesting}
%	\item \href{https://www.cybrary.it/0p3n/heathen-iot-pentesting-framework-released/}{Framework pentesting}
%	\item \href{https://github.com/phodal/awesome-iot}{IoT github}
%\end{itemize}
%
%Currently there are a lot of IoT devices. The next links show differents classifications.
%\begin{itemize}
%	\item \href{http://iotlist.co/}{Here we can find a big ammount of IoT devices}
%	
%	\item \href{https://www.micrium.com/iot/devices/}{Here we can find a company dedicated to proporcionate features for help developers to build microprocessors, microcontroller or DSP-based device and an explanation of how works IoT}
%	\item \href{http://www.pcmag.com/article2/0,2817,2410889,00.asp}{Popular smart home devices}
%\end{itemize}
%	\item \href{https://www.bbvaopenmind.com/7-tendencias-de-internet-de-las-cosas-en-2017/}{no sirve mucho}
%\item \href{https://www.postscapes.com/internet-of-things-award/connected-home-products/}{\IoT most popular}
%\item \href{https://www.postscapes.com/internet-of-things-award/winners/}{\IoT popular 2017}
%http://www.beyondsecurity.com/security_testing_iot_internet_of_things.html
	
	\newpage
	
	\section{IoT classification}
	\label{sec:iot}
	%iot

In this section there is a classification of the different types of IoT, according to the similar characteristics that the majority of them have in common (scope of activity, target, data, ways of communication, etc.) They are ordered from the biggest market with wide potencial, to the smallest market. \\
This classification help to identify the fields in which the IoT are developed, and the size of the markets that exist right now. This is usefull if we want to launch a new product and we want to know what kind of advertising and strategies are been used in each field, or to test any kind of product and be able to study the features shared.

\subsection{Smart home} A smart home, or smart house, is a home that incorporates advanced automation systems to provide the inhabitants with sophisticated monitoring and control over the building's functions.\\ The Smart Homes have become a popular search in the internet.\\ More companies are active in this scope than any other application in the IoT field. In the list of recognized gadgets are the names of Nest or AlertMe, as well as a number of multinational corporations such as Philips, Haier, or Belkin. The most common features in a smart home are control lighting, temperature, multi-media, security, window and door operations, as well as many other functions.

\subsection{Wearables} The wearables are smart electronic devices that can be worn on the body as implant or accessories. These devices show the most significant daily growth, derived from the large number of emerging terminals that are manufactured and marketed daily. Examples such as the smartwatchs of different companies are emerging every day, as well as bands tracking daily activity, etc.

\subsection{Smart city} A smart city is an urban development vision to integrate information and communication technology to manage a city's assets.\\ For example, manage waste, traffic management, water, urban security and environmental monitoring, making them more efficient and safer, thanks to the Devices and Internet technology of the Things (\ IoT)\\ For these reasons smart cities are searched every day by several users.

\subsection{Smart grid} A smart grid is an electrical grid which includes a variety of operational and energy measures including smart meters, smart appliances, renewable energy resources, and energy efficient resources.\\
A future of smart grids is committed to using information on the behavior of electricity suppliers and consumers in an automated way to improve the efficiency, reliability and economy of electricity.

\subsection{Industrial Internet of Things (IIoT)} The Industrial Internet of Things (IIoT) is the use of Internet Technologies of Things (IoT) in manufacturing.\\ Many market studies such as Gartner or Cisco see in the Internet Industry as the IoT concept with the greatest global potential.

\subsection{Connected car} A connected car is a car that is equipped with Internet access, and usually also with a wireless local area network. This kind of car provides tools to the driver and to the passangers like gps, movies, temperature control, etc... In some years there will be autonomous car that drive by themselves.

\subsection{Health} In this field are several devices that has great potencial. These gadgets can help doctors with monitoring pacients and search in the registry to deseases with those syntoms or warn to the near hospital when a monitoring pacient has an attack.

\subsection{Others} Exists other market for \IoT like agriculture, marketing, advertisment, etc... The differents \IoT widget are useful but less common that above list.
	
	\newpage
	
	\section{Conclusion}
	\label{sec:artcomp}
	%Conclusion

After reading and summarizing all information I come to the following conclusions:

\begin{itemize}
	\item \textbf{Articles:} 
	Los dispositivos IoT tienen fallas de seguridad graves, porque la mayoría busca la funcionalidad y no se han percatado o querido tener en cuenta las medidas necesarias para que la información valiosa de los nuevos dispositivos esté segura. Muchos papers proponen diferentes medidas para desarrollar productos. Pero deben ser las empresas que realizan estas tareas las que deben seguirlos. Uno de los papers aquí expuestos, propone un método para realizarlo una vez el dispositivo esté en el mercado, pudiendo evitar conectarlo si no pasa las pruebas. Creo que este último paper es de suma importancia, porque no solo podemos utilizarlo nosotros como bien dice, sino routers companies para que venga integrado en el proceso de conexión que tenemos en casa. También podrá ser útil para las diferentes empresas que usan dispositivos IoT como herramientas de trabajo.\\
	Otros de los papers expuestos hablan de testing, y de las diferentes formas que han usado para realizar sus pruebas. De los cuales podemos aprender técnicas y realizar los mismos en nuestro trabajo fin de máster.
	
	The majority of the articles refer to differents methodologies used to test IoT environments, however some articles propose some tools and certificates for IoT testing vulnerabilities. It is very interesting to know all the available methodologies and learn the tools for testing.
	
	\item \textbf{Companies:} I divided them in two groups, because I think that in order to protect IoT device it is important doing it in the two differents ways at the same time. Para nuestro trabajo fin de máster sería conveniente hablar con las empresas de routers para introducir nuestro trabajo en ellos de forma nativa y sea eficiente a la hora de evaluar si un dispositivo puede unirse a la red o no.

	\item \textbf{Pen-testing:} En esta sección hemos aprendido el concepto de hacking ético, las vulnerabilidades más importantes y los diferentes ataques a realizar
	
	\item \textbf{IoT Classification:} I think the biggest market and the most unprotected is smart house. because when a company get a new device, has the required resources to securite it.
\end{itemize}
	
	\newpage
	
	\bibliographystyle{ieeetr}	
	\begin{thebibliography}{9}	
		
		\bibitem{Paper0}
		Craigen, D., Diakun-Thibault, N., \& Purse, R. (2014). Defining cybersecurity. Technology Innovation Management Review, 4(10).
		
		\bibitem{Paper}
		Amoroso, E. 2006. Cyber Security. New Jersey: Silicon Press.
		
		\bibitem{IoT1}
		S. Haller S. Karnouskos and C. Schroth "The Internet of Things in an Enterprise Context " in Future Internet-FIS 2008 Lecture Notes in Computer Science Vol. 5468 2009 pp 14-28.
		
		\bibitem{IoT2}
		A. C. Sarma and J. Girão "Identities in the Future Internet of Things " in Wireless Personal Communications 49.3 2009 pp. 353-363.
		
		\bibitem{IoT3}
		Roberto Minerva Abiy Biru "Towards a Definition of the Internet of Things " IEEE IoT Initiative white paper.
		
		\bibitem{Paper1} 		
		Jonathan Charity Talwana, Huang Jian Hua, \href{http://ieeexplore.ieee.org/document/7917092/}{Smart World of Internet of Things (IoT) and Its Security Concerns} (15-18 Dec. 2016) 4 May 2017
		
		\bibitem{Data1}
		Gubbi, J., Buyya, R., Marusic, S., \& Palaniswami, M. (2013). Internet of Things (IoT): A vision, architectural elements, and future directions. Future generation computer systems, 29(7), 1645-1660.
		
		\bibitem{Paper2} 
		Eike Steffen Reetz, Daniel Kuemper, Klaus Moessner, \href{http://ieeexplore.ieee.org/document/6582811/}{How to Test IoT-based Services before Deploying them into Real World} (16-18 April 2013) Online - 20 August 2013
		
		\bibitem{Paper3} 
		Pablo Gimenez, Benjamin Molina, Carlos E. Palau, \href{http://ieeexplore.ieee.org/document/6721820/}{SWE Simulation and Testing for the IoT} (13-16 Octubre 2013) 27 Enero 2014
		
		\bibitem{Paper8}
		Lake, D., Milito, R., Morrow, M., \& Vargheese, R. (2014). Internet of things: Architectural framework for ehealth security. Journal of ICT Standardization, River Publishing, 1.
		
		\bibitem{Presentation1}
		Miessler, D. (2015). IoT Attack Surface Mapping.
		
		\bibitem{Paper4} 
		Erik Jan Marinissen, Yervant Zorian, Mario Konijnenburg, \href{http://ieeexplore.ieee.org/document/7519331/}{IoT: Source of test challenges} (23-27 Mayo 2016) 25 Julio 2016
		
		\bibitem{Paper5} 
		Umar Ahsan, Abdul Bais, \href{http://ieeexplore.ieee.org/stamp/stamp.jsp?arnumber=7815042&tag=1}{A Review on Big Data Analysis and Internet of Things} (10-13 Octubre 2016) Publicado el 16 de Enero 2017
		
		\bibitem{Paper6} 
		Salvador Pérez, Juan A. Martínez, Antonio F. Skarmeta, Márcio Mateus, Bruno Almeida, Pedro Maló, \href{http://ieeexplore.ieee.org/document/7845504/}{ARMOUR: Large-scale experiments for IoT security \& trust}
		(12-14 Dec. 2016) publicado el 9 de febrero 2017
		
		\bibitem{Paper7} 
		Ruth M. Ogunnaike; Brent Lagesse, \href{http://ieeexplore.ieee.org/document/7917633/}{Toward consumer-friendly security in smart environments}, (13-17 March 2017), 4 May 2017
		
		\bibitem{ethicalHacking}
		Saleem, S. A. (2006, September). Ethical hacking as a risk management technique. In Proceedings of the 3rd annual conference on Information security curriculum development (pp. 201-203). ACM.
		
		\bibitem{Hacking1}
	    Julia Allen, Alan Christie, William Fithen, John ?McHugh, Jed Pickel, Ed Stoner. (2000). "State of the Practice of Intrusion Detection Technologies". http://www.sei.cmu.edu/publications/documents/99.reports/99tr028/99tr028title.html.
	    
	    \bibitem{Hacking2}
        "NSA Glossary of Terms Used in Security and Intrusion Detection". http://www.sans.org/newlook/resources/glossary.htm.
        
        \bibitem{Hacking3}
         Michael Gregg, Certified Ethical Hacker
	\end{thebibliography}
	
\end{document}